\documentclass{beamer}
\usetheme{default}

\usepackage{caption}
\usepackage{subcaption}

% width of figures
\newcommand\w{0.32}

\title{Simulated annealing results}
\author{T.\ Northey}
\date{Oct 2022}
\begin{document}
\begin{frame}[plain]
    \maketitle
\end{frame}

\begin{frame}
\begin{columns}
	\begin{column}{0.5\textwidth}
		%\begin{itemize}
		The molecular coordinates move iteratively according to,
		\[
			\textbf{R}_{i+1} = \textbf{R}_{i} + \Delta s\sum_{k=0}^{\textrm{modes}} \Big(\frac{\omega_0}{\omega_k}\Big) a_k\hat{d}_k
		\]
		for step-size $\Delta s$, displacement unit vectors $\hat{d}_k$, and wavenumbers $\omega_k$ for each normal mode. The factors $a_k$ are obtained from a uniform random distribution with range $[-T, T]$, for temperature $T$.
		%\end{itemize}
	\end{column}
	\begin{column}{0.5\textwidth}
		The temperature decreases linearly at iteration $i$ as,
		\[
		%T_i = T_0\exp(-\gamma i / N)
			T_i = T_0(1 - i / N)
		\]
		for starting temperature $T_0\in(0, 1]$, and total iterations $N$.

		After each iteration, if the error function $\chi^2$ decreases the iteration is accepted
		\[
		i\rightarrow i+1
		\]
		If not, the iteration can still be accepted with probability,
		\[
		P = T_i
		\]
	\end{column}
\end{columns}
\end{frame}

\begin{frame}{Noise}
\begin{columns}
	\begin{column}{0.5\textwidth}
		The noise is generated at the $j$th grid value of $q$ by another uniform random distribution,
		\[
			r(q_j) = 1 - [-\eta, \eta]_j
		\]
		for noise parameter $\eta\in[0, 1]$, and then the target percent difference $\%\Delta I(q)$ is multiplied by $r(q)$. Example, if $\eta = 0.1$, then each point is muliplied by a random factor in the range $[0.9, 1.1]$.
	\end{column}
	\begin{column}{0.5\textwidth}

	\end{column}
\end{columns}
\end{frame}

