\documentclass[]{article}
\usepackage[margin=3cm]{geometry}
\usepackage{graphicx}
\usepackage{subcaption}
\usepackage{float}
\usepackage{amsmath}
\usepackage{enumerate}
\usepackage{amsmath}
\usepackage{bm}
\usepackage[linesnumbered,ruled,vlined]{algorithm2e}
% custom commands %
\newcommand{\half}{\frac{1}{2}}
\newcommand{\dd}{\textnormal{d}}
\newcommand{\ddx}{\frac{\dd}{\dd x}}
\newcommand{\dydx}{\frac{\dd y}{\dd x}}
\newcommand{\bfx}{\textbf{x}}
\newcommand{\bfR}{\textbf{R}}
%%% redefine eqnarray to not put equation numbering
\newenvironment{eq}
{\begin{eqnarray*}}
	{\end{eqnarray*}}
%%%

%opening
\title{A general simulated annealing approach to extracting nuclear dynamics from ultrafast x-ray scattering data}
\author{Thomas Northey}
\date{\today}

\begin{document}
	
	\maketitle
	
	\begin{abstract}
		abst
	\end{abstract}
	
	\section{Introduction}
	1m structure method \cite{yong2021determination}
	
	Some references \cite{yong2019scattering,yong2021determination,stankus2019ultrafast,wolf2019photochemical,moreno2019ab,northey2014ab,northey2016elastic}
	
	\section{Method}
	
	The molecular coordinates move iteratively according to,
	\[
	\textbf{R}_{i+1} = \textbf{R}_{i} + T_i\Delta s\sum_{k=0}^{\textrm{modes}} \Big(\frac{\omega_0}{\omega_k}\Big) a_k\hat{d}_k
	\]
	for temperature $T_i$ at iteration $i$, step-size $\Delta s$, displacement unit vectors $\hat{d}_k$, and wavenumbers $\omega_k$ for each normal mode. The factors $a_k$ are obtained from a uniform random distribution with range $[-1, 1]$ to allow the molecule unconstrained movement along all its degrees of freedom.
	The motions are $\omega$-damped by the factor (${\omega_0}/{\omega_k}$) to avoid oversampling large motions of high frequency modes.
	
	The temperature decreases linearly at iteration $i$ as,
	\[
	%T_i = T_0\exp(-\gamma i / N)
	T_i = T_0(1 - i / N)
	\]
	for starting temperature $T_0\in(0, 1]$, and total iterations $N$.
	
	After each iteration, if the error function $\chi^2$ decreases the iteration is accepted
	\[
	i\rightarrow i+1
	\]
	If not, the iteration can still be accepted with probability,
	\[
	P = T_i
	\]
	which allows the molecule to sometimes travel uphill on the $\chi^2$ surface, thus escape local minima.
	
	\section{Algorithm}
	
	\subsection{current testing version}
	
	Assumption: the next best fit is `nearby' the previous step.
	
	\begin{itemize}
		\item Start at $\bfR(t_0)$, e.g.\ the optimised ground state geometry
		\begin{itemize}
			\item Start at $\bfR(t_j)$ and search for $\bfR(t_{j+1})$
			\item Perform $N$ temperature cycles ($T = T_0$), to allow a large search space around $\bfR(t_j)$
			\begin{itemize}
				\item Start at the end point of each cycle $\bfR_i$ to find the next cycle end point $\bfR_{i+1}$
				\item Save $\bfR_\textrm{best}$, the geometry with the lowest value of $\chi^2$
				
			\end{itemize}
		\end{itemize}
		
	\end{itemize}
	
	
	\section{Results}

	\begin{figure}[H]
	\centering
	\includegraphics[width=0.8\textwidth]{lineouts.png}
	\caption{CHD x-ray scattering lineouts.}
	\end{figure}
	
	It seems like $\chi^2 \sim 0.001$ corresponds with $R_{CC}$ matching the target.
	
	\begin{figure}[H]
		\centering
		\begin{subfigure}{0.8\textwidth}
		\includegraphics[width=\textwidth]{rcc_plots.png}
		\caption{Temperature cycle method: Do not restart from the starting geometry at each cycle, but start at the last geometry of the previous cycle.}
		\end{subfigure}
	\hfill
		\begin{subfigure}{0.8\textwidth}
			\includegraphics[width=\textwidth]{rcc_plots.png}
			\caption{Robust restart method: Restart from a slight perturbation of the starting geometry at each cycle, and use 5x the amount of steps per cycle.}
		\end{subfigure}
		\caption{CHD ring-opening C-C distance, $R_{CC}$. The target is from a surface hopping trajectory. Predictions are using $q_\textrm{max} = 25.0$ \AA$^{-1}$.}
		\label{fig:rcc_q25}
	\end{figure}

	\begin{figure}[H]
	\centering

		\begin{subfigure}{0.6\textwidth}
			\includegraphics[width=\textwidth]{target_step48.png}
			\caption{Target.}
			\label{fig:chd_target}
		\end{subfigure}
	\hfill
		\begin{subfigure}{0.6\textwidth}
			\includegraphics[width=\textwidth]{1003.png}
			\caption{Predicted.}
			\label{fig:chd_predicted}
		\end{subfigure}
	\caption{CHD target and predicted, showing all C-C bond-lengths.}
	\label{fig:figures}
	\end{figure}

\bibliographystyle{plain}
\bibliography{library}
	
\end{document}
